\documentclass[12pt]{book}
\usepackage[cp1251]{inputenc}
\usepackage[T2A]{fontenc}
\usepackage[T1]{fontenc}
\usepackage[english, russian]{babel}
\usepackage[colorlinks=true, allcolors=blue]{hyperref}
\usepackage[colorinlistoftodos]{todonotes}
\usepackage[fleqn]{amsmath}
\usepackage{amssymb}

\usepackage{color}
\definecolor{light-gray}{rgb}{0.8,0.8,0.8}
\setcounter{chapter}{4}
\setcounter{section}{3}
\setcounter{subsection}{4}
\setcounter{equation}{37}

\setcounter{page}{69}

\begin{document}
Each fraction has the form $1+x$ with $x=\pm 1\!/3$. Because x is small, one
term of the logarithm series might provide reasonable accuracy. Let's therefore use $\ln(1+x)\approx x$ to approximate the two logarithms:
\begin{equation}
\ln2\approx \frac 13 - (-\frac 13)=\frac 23.
\end{equation}
This estimate is accurate to within 5\%!
The rewriting trick has helped to compute $\pi$ (by rewriting the $\arctan{x}$ series) and to estimate $\ln(1+x)$ (by rewriting $x$ itself). This idea therefore becomes a method --- a trick that I use twice (this definition is often attributed to Polya).\\

\colorbox{light-gray}{
\begin{minipage}{\textwidth}
{\bf Problem 4.20 How many terms?}\\
The full Taylor series for the logarithm is
\begin{equation}
\ln(1+x)=\sum_{1}^\infty (-1)^{n+1} \frac {x^{n}}n
\end{equation}

If you set $x=1$ in this series, how many terms are required to estimate $\ln2$ to within 5\%?\\

{\bf Problem 4.21 Second rewriting}\\
Repeat the rewriting method by rewriting $ 4\!/3$ and $ 2\!/3$; then estimate $\ln2$ using only one term of the logarithm series. How accurate is the revised estimate?\\

{\bf Problem 4.22 Two terms of the Taylor series}\\

After rewriting $\ln2$ as $\ln(4\!/3) - \ln(2\!/3)$, use the two-term approximation that $\ln(1+x)\approx x-x^2\!/2$ to estimate $\ln2$ Compare the approximation to the one-term estimate, namely $ 2\!/3$. (Problem 4.24 investigates a pictorial explanation.)\\

{\bf Problem 4.23 Rational-function approximation for the logarithm}\\
The replacement $\ln2=\ln(4\!/3) - \ln(2\!/3)$ has the general form
\begin{equation}
\ln(1+x)=\ln \frac{1+y}{1-y},
\end{equation}
where $y=x\!/(2+x)$.
\end{minipage}
}
\colorbox{light-gray}{
\begin{minipage}{\textwidth}
Use the expression for $y$ and the one-term series $\ln(1+x)\approx x$ to express $\ln(1+x)$ as a rational function of $x$ (as a ratio of polynomials in $x$). What are the first few terms of its Taylor series?
Compare those terms to the first few terms of the $\ln(1+x)$ Taylor series, and thereby explain why the rational-function approximation is more accurate than even the two-term series $\ln(1+x)\approx x-x^2\!/2$.\\

{\bf Problem 4.24 Pictorial interpretation of the rewriting}\\
\renewcommand{\labelenumi}{\alph{enumi})}
\begin{enumerate}
\item Use the integral representation of $\ln(1+x)$ to explain why the shaded area is $\ln2$.
\item Outline the region that represents
\begin{equation}
\ln\frac 43 - \ln\frac 23
\end{equation}
when using the circumscribed-rectangle approximation for each logarithm.
\item Outline the same region when using the trapezoid approximation\\ $\ln(1+x)\approx x-x^2\!/2$. Show pictorially that this region, although a different shape, has the same area as the region that you drew in item b.
\end{enumerate}
\end{minipage}
}
\section{ Bisecting a triangle}
Pictorial solutions are especially likely for a geometric problem:\\

\noindent{\it What is the shortest path that bisects an equilateral triangle into two regions of equal area?}\\

\noindent The possible bisecting paths form an uncountably infinite set. To manage the complexity, try easy cases (Chapter 2)—draw a few equilateral triangles and bisect them with easy paths. Patterns, ideas, or even a solution might emerge.\\

\noindent{\it What are a few easy paths?}\\

\noindent The simplest bisecting path is a vertical segment that splits the triangle into two right triangles each with base $1\!/2$. This
path is the triangle's altitude, and it has length
\begin{equation}
l=\sqrt{1^{2}-(1\!/2)^{2}}=\frac{\sqrt{3}}{2} \approx 0.866.
\end{equation}
An alternative straight path splits the triangle into a trapezoid and a small triangle.\\

\noindent{\it What is the shape of the smaller triangle, and how long is the path?}\\

\noindent The triangle is similar to the original triangle, so it too is equilateral.\\Furthermore, it has one-half of the area of the original triangle, so its three sides, one of which is the bisecting path, are a factor of $\sqrt{2}$ smaller than the sides of the original triangle. Thus this path has length $1\!/\sqrt{2} \approx 0/707$ -- a substantial improvement on the vertical path with length $\sqrt{3}\!/2$.\\

\colorbox{light-gray}{
\begin{minipage}{\textwidth}
{\bf Problem 4.25 All one-segment paths}\\
An equilateral triangle has infinitely many one-segment bisecting paths. A few of them are shown in the figure. Which one-segment path is the shortest?
\end{minipage}
}\\

\noindent Now lets investigate easy two-segment paths. One possible path encloses a diamond and excludes two small triangles. The two small triangles occupy one-half of the entire area. Each small triangle therefore occupies one-fourth of the entire area and has side length $1\!/2$. Because the bisecting path contains two of these sides, it has length 1. This path is, unfortunately, longer than our two one-segment candidates, whose lengths are $1\!/\sqrt{2}$ and $\sqrt{3}\!/2$. Therefore, a reasonable conjecture is that the shortest path has the fewest segments. This conjecture deserves to be tested (Problem 4.26).\\

\colorbox{light-gray}{
\begin{minipage}{\textwidth}
{\bf Problem 4.26 All two-segment paths}\\
Draw a figure showing the variety of two-segment paths. Find the shortest path,
showing that it has length
\begin{equation}
l=2\times 3^{1\!/4}\times \sin 15^{\circ} \approx 0.681.
\end{equation}
{\bf Problem 4.27 Bisecting with closed paths}\\
The bisecting path need not begin or end at an edge of the triangle. Two examples are illustrated here.
\end{minipage}
}\\

\colorbox{light-gray}{
\begin{minipage}{\textwidth}
Do you expect closed bisecting paths to be longer or shorter than the shortest one-segment path? Give a geometric reason for your conjecture, and check the conjecture by finding the lengths of the two illustrative closed paths.
\end{minipage}
}\\

\noindent{\it Does using fewer segments produce shorter paths?}\\

\noindent The shortest one-segment path has an approximate length of $0.707$; but the shortest two-segment path has an approximate length of $0.681$.The length decrease suggests trying extreme paths: paths with an infinite number of segments. In other words, try curved paths. The easiest curved path is probably a circle or a piece of a circle.\\

\noindent{\it What is a likely candidate for the shortest circle or piece of a circle that bisects the triangle?}\\

\noindent Whether the path is a circle or piece of a circle, it needs a center.However, putting the center inside the triangle and using a full circle produces a long bisecting path (Problem 4.27). The only other plausible center is a vertex of the triangle, so imagine a
bisecting arc centered on one vertex.\\

\noindent{\it How long is this arc?}\\

\noindent The arc subtends one-sixth $(60^{\circ})$ of the full circle, so its length is $l=\pi r\!/3$, where $r$ r is radius of the full circle. To find the radius, use the requirement that the arc must bisect the triangle. Therefore, the arc encloses one-half
of the triangle's area. The condition on $r$ is that $\pi r^2=3\sqrt{3}\!/4$:
\begin{equation}
\frac 16 \times \underbrace{area~of~the~full~circle}_{\pi r^2}=\frac 12 \times \underbrace{area~of~the~triangle}_{\sqrt{3}\!/4}.
\end{equation}
\noindent The radius is therefore $(3\sqrt{3}\!/4\pi)^{1\!/2}$; the length of the arc is $\pi r\!/3$, which is approximately $0.673$. 
This curved path is shorter than the shortest two-segment path. It might be the shortest possible path.\\

\noindent To test this conjecture, we use symmetry. Because an equilateral triangle is one-sixth of a hexagon, build a hexagon by replicating the bisected equilateral triangle. Here is the hexagon built from the triangle bisected by a horizontal line.\\

\noindent The six bisecting paths form an internal hexagon whose area is one-half of the area of the large hexagon.\\

\noindent{\it What happens when replicating the triangle bisected by the circular arc?}\\
\end{document} 

